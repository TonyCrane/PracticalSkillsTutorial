\documentclass{article}
\usepackage{ctex}
\usepackage[draft]{graphicx}
\usepackage{enumerate}
\usepackage{amsmath}

\title{第四讲:\LaTeX{}排版简要介绍}
\author{TonyCrane\thanks{me@tonycrane.cc}\and 鹤翔万里\thanks{也是我}}
\date{\today}

\begin{document}

\maketitle

\tableofcontents

\listoftables

\section{段落}\label{sec:para}

Several spaces     equal one.
   Front spaces are ignored. 中文结尾
不会添加空格。

空行可以换段落。\par
或者使用 \verb|\par| 也可以。

This is a paragraph, and I want to break it. \\
This is the next line, but it is not indented.

This is a new paragraph.
This is the next page.

\% \& \# \{ \} \~{} \^{} \textbackslash

`sing quote' ``double quote''

difficult dif{}f{}icult

\subsection{字体字号}

\textbf{bold} \textit{italic} \texttt{typewriter}
\textsf{sans serif} \textsc{Small Caps} \textsl{slanted}

{\tiny tiny} {\scriptsize scriptsize} {\footnotesize footnotesize}
{\small small} {\normalsize normalsize} {\large large}
{\Large Large} {\LARGE LARGE} {\huge huge} {\Huge Huge}

{\linespread{2.0}\selectfont This is a paragraph with 2.0 linespread.\\
This is the next line and \quad space and \hspace{3em} more and
\hspace{\fill}lol\par}

\vspace{-1em}Next paragraph.\\[1em] and next line.

\section{文档元素}

some text\footnote{footnote}

\begin{tabular}{l}
\hline some text\footnotemark \\\hline
\end{tabular}
\footnotetext{table footnote}

\includegraphics[width=10em,angle=30]{avatar}

\begin{itemize}
    \item First item
    \item[+] Second item
    \begin{enumerate}[i.]
        % \setcounter{enumi}{2}
        \item First subitem\label{item:subitem}
        \item Second subitem
    \end{enumerate}
\end{itemize}

\begin{description}
    \item[First] First item
    \item[Second] Second item
\end{description}

\begin{center}
some text
\end{center}
\begin{flushright}some text\end{flushright}

\centering some text

\raggedleft some text
\raggedright

\begin{verbatim}
#include <stdio.h>
int main() {
    printf("Hello, world!\n");
    return 0;
}
\end{verbatim}

\verb|\LaTeX| and \verb*|printf("Hello, world!\n");|

\begin{tabular}{@{}c@{:}c|c}
a & b & c \\
\cline{2-3}
d & e & f \\
\end{tabular}

\begin{table}
\caption{TablesGenerator}\label{tab:table}
\begin{tabular}{|lllll}
\hline
\multicolumn{1}{|l|}{a} & \multicolumn{1}{l|}{b} & \multicolumn{1}{l|}{} & \multicolumn{1}{l|}{} & \multicolumn{1}{l|}{} \\ \hline
\multicolumn{1}{|l|}{}  & \multicolumn{2}{l|}{c}                         & \multicolumn{1}{l|}{} & \multicolumn{1}{l|}{} \\ \hline
\multicolumn{1}{|l|}{}  & \multicolumn{1}{l|}{}  &                       &                       &                       \\ \hline
                        & \multicolumn{1}{l|}{}  & \multicolumn{1}{l|}{} & \multicolumn{1}{l|}{} & \multicolumn{1}{l|}{} \\ \hline
\end{tabular}
\end{table}

请见第\pageref{tab:table}页中的表\ref{tab:table}。本文第一个章节编号为\ref{sec:para}。\ref{item:subitem}

\appendix

\newpage

\section{数学公式}

行内公式:$a^2+b^2=c^2$、$\text{text}$、$no space$。

行间公式:
\begin{equation}
    \int_{-\infty}^{+\infty} e^{-x^2} dx = \sqrt{\pi}
\end{equation}
$$
a^2+b^2=c^2
$$
\[a^2+b^2=c^2\]

\begin{displaymath}
    a^2+b^2=c^2
\end{displaymath}

$a\,b$、$a\:b$、$a\;b$、$a\quad b$、$a\qquad b$、$a\!b$

$a^b$、$a^{bc}$、$a^{b^c}$、$a_b$、$a_{bc}$、$a_{b_c}$、$a^b_c$、$\sum_{i=1}^{10}...$、$\int_a^b$、$\frac{1}{2}$

$$\sum_{i=1}^{10}...\int_a^b\frac{1}{2}$$

$\sin(\pi)$

$\displaystyle\sum_{i=1}^{10}...\int_a^b\frac{1}{2}$、$\displaystyle\frac{1}{2}$、$\dfrac12$

$$
(abc), (\frac{1}{2}), \left(\frac{1}{2}\right), \left(\frac{1}{2}\right.
$$

\begin{align}
a &= b + c  &     g &= h + i \\
  &= d + e + f  &   &= j\notag
\end{align}

\begin{equation}
\begin{aligned}
a &= b + c  &     g &= h + i \\
  &= d + e + f  &   &= j
\end{aligned}
\end{equation}

\[
\mathbf{A} = \begin{vmatrix}
a_{11} & a_{12} & \cdots & a_{1n}\\
a_{21} & a_{22} & \cdots & a_{2n}\\
\vdots & \vdots & \ddots & \vdots\\
a_{n1} & a_{n2} & \cdots & a_{nn}\\
\end{vmatrix}
\]

\end{document}